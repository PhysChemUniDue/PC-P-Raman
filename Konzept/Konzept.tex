% !TEX root = TeX/Raman.tex
\section*{PC-P: Raman-Mikroskopie}

\subsection*{Was wird am Versuchstag gemacht?}

\subsubsection*{Probenvorbereitung:} 
    \begin{itemize}
        \item Metalle, wie Eisen, Kupfer oder Nickel werden gezielt korrodiert.
         z.B. durch Behandlung mit Salzwasser, Essigs"aure, Ammoniak oder \ce{H2O2}.
    \end{itemize}
\subsubsection*{Raman-Spektroskopie:}
    \begin{itemize}
        \item Kalibration des Raman-Mikroskops. (z.B. mit Polystyrol-Standard)
        \item Licht-Mikroskopie: Identifikation interessanter Zonen (korrodiert vs. blank).
        \item Aufnahme einzelner Raman-Spektren an charakteristischen Punkten der interessanten Zonen
        \item Raman-Mapping "uber Bereiche mit unterschiedlicher Zusammensetzung.
    \end{itemize}


\subsection*{Was wird nach dem Versuch ausgewertet?}

\begin{itemize}
    \item Data Handling: Hintergrundkorrektur, Gl"attung, Normalisierung
    \item Peak-Identifikation und Zuordnung zu chemischen Phasen (z.B. \ce{Fe2O3}, \ce{Fe3O4}, \ce{CuO})
    \item Vergleich mit Mikroskopie-Bildern zur morphologischen Interpretation
\end{itemize}

\subsection*{Welche Proben eignen sich?}

\begin{itemize}
    \item Korrodiertes Eisen (z.B. N"agel, Schrauben) oder doch besser blankes Eisen?
    \item Kupferm"unzen?
    \item beschriftetes Papier (Graphit)?
\end{itemize}


